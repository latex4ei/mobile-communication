% % % % % % % % % % % % % % % % % % % % % % % % % % % % % % % % % % % % % % % % 
% Formelsammlung von LaTeX4EI									
%
% @encode: 	UTF-8, tabwidth = 4, newline = LF
% @author:	Emanuel Regnath
%
% % % % % % % % % % % % % % % % % % % % % % % % % % % % % % % % % % % % % % % % 


% .:: Laden der LaTeX4EI Formelsammlungsvorlage
\documentclass[fs, footer]{latex4ei}


%---------------------------------------%
%			Mobile Communication		%
%~~~~~~~~~~~~~~~~~~~~~~~~~~~~~~~~~~~~~~~%

% DOCUMENT_BEGIN ===============================================================
\begin{document}

% Split in 4 Columns ===========================================================
\begin{multicols*}{4}

% TITLE ========================================================================
\fstitle{Mobile\\ Communication}


% SECTION ====================================================================================
\section{Overview}
% ============================================================================================

LOS = Line of Sight: Path with high Amplitude

\sectionbox{

Broadcast: Same message to all clients\\
Unicast: individual data\\

unidirectional or bidirectionaal communication.

service quality: data rate, delay, error rate\\

ALOHA System: User sends packet whenever the data source makes them available.\\
Carrier-Sense Multiple-Access: If channel is busy, the user waits a random time until its transmission

}
	\subsection{UMTS}
	Base station $\SI{5}{\watt}$ TX-Power\\
	Mobile $\SI{0.125}{\watt}$ TX-Power

	\subsection{Cellular Principle}
	macro cells (global), micro cells (local), pico cells (small room)\\
	Typical cluster size in GSM is 7


	\subsection{Distorsions within the channel}
	\begin{description}
		\item[Shadowing („Abschattung“)] because of large obstacles
		\item[Reflection („Reflexion“)] on large on areas
		\item[Refraction (“Brechung”)] on surfaces \\ (change of medium $\ra$ change of speed)
		\item[Scattering („Streuung“)] at small obstacles
		\item[Diffraction („Beugung“)] on sharp edges
		\item[Interference]
	\end{description}

	Noise: radio noise in the air, thermal noise of wires,


	Friis free space equation:\\
	$P_{\ir RX}(d) = \frac{P_{\ir TX} G_{\ir RX} G_{\ir TX} \lambda^2}{(4 \pi d)^2 L}$\\



	Path loss: $L_p(d) = L_{p0} + 10 \alpha \log_{10}(d)$
	
	\subsection{Signal Transmission}
	LowPASS Equivalent: $s(t) = A(t) \cdot e^{\i \varphi (t)}$\\
	BandPASS Signal: $u(t) = \Re\{s(t) e^{\j 2\pi f t} \}$ \qquad f:Carrierfreq.\\
	Transmitted Signal: $u(t) = A(t) \cos(2\pi ft + \phi(t))$\\
	Received Signal: $r(t) = \frac{\alpha(\theta,\Psi,f) \cos(2\pi f(t - \frac{d}{c}))}{d}$\\

	
	Doppler Shift $D_f = -f \frac{v}{c} \cos(\gamma)$\\


	\begin{tabular}{l|l}
		static & moving\\
		diffrent delays of paths & diffrent doppler frequencies\\
		frequency selective & time selective\\[0.5em]
		Delay spread & Doppler spread:\\
		$T_d = \tau_{\max} - \tau_{\min}$ & $f_{\ir DS} = f_{\ir D,max} - f_{\ir D,min}$\\[0.5em]
		Coherence Bandwidth: & Coherence Time:\\
		
	
	
	\end{tabular}


	\subsection{Parameters of a Channel}
	\begin{tabular}{lll}
	carrier frequency & $f_{\ir c}$ & $1 - \SI{2}{\giga\hertz}$ \\
	used bandwidth & $W = \frac{1}{T_{\ir T}}$ & 0.2 – 10 MHz\\
	distance between transmitter and receiver & $D$ & $\SI{1}{\kilo\meter}$\\
	Velocity of mobile & $v$ & $\SI{50}{\kilo\meter\per\hour}$\\
	Time-scale for change of path amplitude dcor/v >3 s\\
	(shadowing, path loss) \\
	Time-scale for a path to move over a tap $T_{\ir S} \cdot \frac{v}{c}$ & $>\SI{3}{second}$\\
	maximum Doppler shift for one path fD=fcv/c 50 – 100 Hz\\
	Doppler spread of paths corresponding to one tap fDS 100 – 200 Hz\\
	Time-scale for change of path phase 1/fD 10 – 20 ms\\
	Coherence time Tcoh= 1/fDS 5 – 10 ms\\
	Delay spread & $T_d$ & $\SI{1}{\micro\second}$\\
	Coherence bandwidth fcoh=1/Td 1 MHz\\
	\end{tabular}
	
	

	\subsection{Deterministic LTV Channel Model}
	% \tau = Delay
	Channel Response: $r(t) = \int\limits_{-\infty}^\infty s(t-\tau) h(\tau,t) \diff \tau$\\
	Ellipsoid Model: Paths with same delay are grouped together.



	\subsection{Modulation Techniques}
	Amplitude Shift Keying (ASK, AM)\\
	Frequency Shift Keying (FSK, FM)\\
	Phase Shift Kesying (PSK, PM)\\
	
	


	\subsection{Multiple Access Techniques}
	FDMA
	TDMA
	CDMA


	\begin{tabular}{ll}
	g & $\si{\dB}$\\
	1 & +0\\
	1.25 & +1\\
	1.6 & +2\\
	2 & +3\\
	\end{tabular}

% Ende der Spalten
\end{multicols*}

% Dokumentende
% ======================================================================
\end{document}

% ToDos:

